%%%%%%%%%%%%%%%%%%%%%%%%%%%%%%%%%%%%%%%%%
% baposter Landscape Poster
% LaTeX Template
% Version 1.0 (11/06/13)
%
% baposter Class Created by:
% Brian Amberg (baposter@brian-amberg.de)
%
% This template has been downloaded from:
% http://www.LaTeXTemplates.com
%
% License:
% CC BY-NC-SA 3.0 (http://creativecommons.org/licenses/by-nc-sa/3.0/)
%
%%%%%%%%%%%%%%%%%%%%%%%%%%%%%%%%%%%%%%%%%

%----------------------------------------------------------------------------------------
%	PACKAGES AND OTHER DOCUMENT CONFIGURATIONS
%----------------------------------------------------------------------------------------

\documentclass[portrait,a0paper,fontscale=0.285]{baposter} % Adjust the font scale/size here

\usepackage{graphicx} % Required for including images
\graphicspath{{figures/}} % Directory in which figures are stored

\usepackage{amsmath} % For typesetting math
\usepackage{amssymb} % Adds new symbols to be used in math mode

\usepackage{booktabs} % Top and bottom rules for tables
\usepackage{enumitem} % Used to reduce itemize/enumerate spacing
\usepackage{palatino} % Use the Palatino font
\usepackage[font=small,labelfont=bf]{caption} % Required for specifying captions to tables and figures

\usepackage{multicol} % Required for multiple columns
\setlength{\columnsep}{1.5em} % Slightly increase the space between columns
\setlength{\columnseprule}{0mm} % No horizontal rule between columns

\usepackage{tikz} % Required for flow chart
\usetikzlibrary{shapes,arrows} % Tikz libraries required for the flow chart in the template

\newcommand{\compresslist}{ % Define a command to reduce spacing within itemize/enumerate environments, this is used right after \begin{itemize} or \begin{enumerate}
\setlength{\itemsep}{1pt}
\setlength{\parskip}{0pt}
\setlength{\parsep}{0pt}
}

\definecolor{red}{rgb}{0.8,0.2,0.2} % Defines the color used for content box headers

\begin{document}

\begin{poster}
{
headerborder=closed, % Adds a border around the header of content boxes
colspacing=1em, % Column spacing
bgColorOne=white, % Background color for the gradient on the left side of the poster
bgColorTwo=white, % Background color for the gradient on the right side of the poster
borderColor=red, % Border color
headerColorOne=black, % Background color for the header in the content boxes (left side)
headerColorTwo=red, % Background color for the header in the content boxes (right side)
headerFontColor=white, % Text color for the header text in the content boxes
boxColorOne=white, % Background color of the content boxes
textborder=roundedleft, % Format of the border around content boxes, can be: none, bars, coils, triangles, rectangle, rounded, roundedsmall, roundedright or faded
eyecatcher=true, % Set to false for ignoring the left logo in the title and move the title left
headerheight=0.1\textheight, % Height of the header
headershape=roundedright, % Specify the rounded corner in the content box headers, can be: rectangle, small-rounded, roundedright, roundedleft or rounded
headerfont=\Large\bf\textsc, % Large, bold and sans serif font in the headers of content boxes
%textfont={\setlength{\parindent}{1.5em}}, % Uncomment for paragraph indentation
linewidth=2pt % Width of the border lines around content boxes
}
%----------------------------------------------------------------------------------------
%	TITLE SECTION 
%----------------------------------------------------------------------------------------
%
{\includegraphics[height=6em]{USC.png}} % First university/lab logo on the left
{\bf\textsc{Scaling blockchain cryptocurrencies with leveled zones}\vspace{0.3em}} % Poster title
{\textsc{\{ Borys Gurtovyi and Eduard Sanou \} \\ University of Southern California}} % Author names and institution
{\includegraphics[height=6em]{USC.png}} % Second university/lab logo on the right

%----------------------------------------------------------------------------------------
%	INTRODUCTION
%----------------------------------------------------------------------------------------

\headerbox{Introduction}{name=introduction,column=0,span=2,row=0}{

\begin{multicols}{2}
\vspace{1em}

Current decentralized cryptocurrencies based on \textbf{blockchains suffer from a
scalability problem} that makes them unfit to replace centralized payment
systems such as VISA and PayPal.

\vspace{1em}
We propose an idea to achieve a higher scalability than current
cryptocurrencies by creating a \textbf{hierarchy of transaction zones at different
levels}, in which the transactions have different properties depending on the
smallest level shared zone.

\textbf{Each level would have a fork of a common blockchain that would be merged with
sibling zones (those under the same parent) periodically}, with the periodicity
being different at each level.  The periodicity would be lower on higher level
zones.

Transactions within the lowest level zones would be applied and verified
inmediately whereas those traversing zones would only be applied once the forks of
the corresponding zones are merged.

We suggest using 4 levels: global (0), country (-1), state (-2), city (-3).

\vspace{1em}
This design would allow the transaction throughput within a single city to
achieve a similar throughput as current blockchains do globally.

In other words, we are \textbf{optimizing transaction delays by distance}, which is a
common expectation of fiat transactions, while allowing the global transaction
throughput to increase significantly.

\end{multicols}

}

%----------------------------------------------------------------------------------------
%	BACKGROUND
%----------------------------------------------------------------------------------------

%\headerbox{Background}{name=results2,column=1,below=objectives,bottomaligned=conclusion}{ % This block's bottom aligns with the bottom of the conclusion block
\headerbox{Background}{name=background,column=2,span=1,row=0}{

Donec faucibus purus at tortor egestas eu fermentum dolor facilisis. Maecenas tempor dui eu neque fringilla rutrum. Mauris \emph{lobortis} nisl accumsan.

Bitcoin, Ethereum, other cryptocurrencies not scalable.  Scalable approaches are bla bla bla, and have reached this throughput bla bla.  Widely used payment systems like Visa and Paypal have higher throughput bla bla.

Nulla ut porttitor enim. Suspendisse venenatis dui eget eros gravida tempor. Mauris feugiat elit et augue placerat ultrices. Morbi accumsan enim nec tortor consectetur non commodo.

\begin{center}
\begin{tabular}{l l}
\toprule
\textbf{Payment system} & \textbf{transactions per second} \\
\midrule
Ethereum & 20 \\
Bitcoin & 3-4 \\
Visa & 24,000 \\
Paypal &  193 \\
\bottomrule
\end{tabular}
\captionof{table}{Table caption}
\end{center}
}

%----------------------------------------------------------------------------------------
%	MAIN
%----------------------------------------------------------------------------------------

\headerbox{Scaling with leveled zones}{name=main,column=0,span=2,row=1,below=introduction,above=bottom}{

\begin{multicols}{2}
\vspace{1em}
\begin{center}
\includegraphics[width=0.8\linewidth]{placeholder}
\captionof{figure}{Figure caption}
\end{center}

The global blockchain is periodically being forked into fine grained local blockchains and periodically being merged into coarser more-global blockchains.

\begin{enumerate}

    \item User assigns a wallet public key to a particular city. This information is stored in the global blockchain. However, we don't restrict him from having multiple accounts in different cities. The locality is called "city" in our system if its population exceeds 1 million people. Other smaller localities will be combined into one fork.

    \item A wallet can claim to be in a new city any time, but this information will propagate to the global (0) blockchain in the same way as transactions.

    \item Every zone creates a fork from the blockchain fork of the upper level periodically.

    \item Each zone only accepts transactions that originate from the same zone.

    \item In-zone transactions are verified during the merge of the lower level zones (or as they are collected into blocks in the city (-3) level). [add graph here]

    \item Out-zone transactions are verified when the lowest common upper level zone merges the zones it entails. [add graph here]

    \item Transactions are verified when the forks they belong to are merged into higher zones (or as they are collected into blocks in the city (-3) level).

\end{enumerate}

4 levels (from high to low) : global (0), country (-1), state (-2), city (-3)

Intervals:
\begin{itemize}
    \item Global (0): 24h
    \item Country (-1): 2h
    \item State (-2): 10 minutes
    \item City (-3): 10 seconds
\end{itemize}

When user 1 (from city A)  sends a cross-city transaction to user 2 (from city B) this transaction is being added to the city A fork as a special  “hold” transaction. Basically, user 1 cannot double-spend this money, however user 2 will not receive them until city-merge will happen.

After merge 
Merges will not have conflicts, allowing them to be fast.  Only verification against double spending is required.

If a user commutes between zones, they can have an account for each zone and split their money among the accounts.

No need to verify the actual location of the user VS the claimed location.  The claimed location only serves to decide in which local fork the account can do transactions.

\begin{center}
\includegraphics[width=0.8\linewidth]{placeholder}
\captionof{figure}{Figure caption}
\end{center}

\end{multicols}
}

%----------------------------------------------------------------------------------------
%	CHALLENGES
%----------------------------------------------------------------------------------------

\headerbox{Challenges}{name=challenges,column=2,span=1,below=background}{

\begin{itemize}[leftmargin=1em]
    \item \textbf{The merging process requires more computational work and bandwidth on
        higher levels}.  Not all nodes may be capable enough to verify merges at
        every level,  Flexibility in the amount of contribution by node may be
        required.
    \item \textbf{When a node performs a merge, it needs to verify the forks of the
        sibling zones for correctness}.  This could require a lot of computation
        resources.  A possible optimization would be that each merge outputs a
        compressed form of the result that makes verification at the next level
        faster.
    \item \textbf{At the lowest level, each zone would have a small number of nodes}.
        This opens the possibility of an attacker disrupting the zone by
        registering nodes in it with high validation weight (either
        computational power or stake).  Such disruption could be a denial of
        service or a denial of selected transactions.
    \begin{itemize}[leftmargin=1em]
        \item A way to make make sure that all zones have the same computing
            power could be useful to solve this.
    \end{itemize}
    \item \textbf{At each level, the merging time between forks should be faster than the
        periodicity of merges}.
\end{itemize}

}

%----------------------------------------------------------------------------------------
%	REFERENCES
%----------------------------------------------------------------------------------------

\headerbox{References}{name=references,column=2,span=1,below=challenges,above=bottom}{

\renewcommand{\section}[2]{\vskip 0.05em} % Get rid of the default "References" section title
\nocite{*} % Insert publications even if they are not cited in the poster
\small{ % Reduce the font size in this block
\bibliographystyle{unsrt}
\bibliography{sample} % Use sample.bib as the bibliography file
}}

\end{poster}

\end{document}
