%%%%%%%%%%%%%%%%%%%%%%%%%%%%%%%%%%%%%%%%%
% baposter Landscape Poster
% LaTeX Template
% Version 1.0 (11/06/13)
%
% baposter Class Created by:
% Brian Amberg (baposter@brian-amberg.de)
%
% This template has been downloaded from:
% http://www.LaTeXTemplates.com
%
% License:
% CC BY-NC-SA 3.0 (http://creativecommons.org/licenses/by-nc-sa/3.0/)
%
%%%%%%%%%%%%%%%%%%%%%%%%%%%%%%%%%%%%%%%%%

%----------------------------------------------------------------------------------------
%	PACKAGES AND OTHER DOCUMENT CONFIGURATIONS
%----------------------------------------------------------------------------------------

\documentclass[portrait,a0paper,fontscale=0.285]{baposter} % Adjust the font scale/size here

\usepackage{graphicx} % Required for including images
\graphicspath{{figures/}} % Directory in which figures are stored

\usepackage{amsmath} % For typesetting math
\usepackage{amssymb} % Adds new symbols to be used in math mode

\usepackage{booktabs} % Top and bottom rules for tables
\usepackage{enumitem} % Used to reduce itemize/enumerate spacing
\usepackage{palatino} % Use the Palatino font
\usepackage[font=small,labelfont=bf]{caption} % Required for specifying captions to tables and figures

\usepackage{multicol} % Required for multiple columns
\setlength{\columnsep}{1.5em} % Slightly increase the space between columns
\setlength{\columnseprule}{0mm} % No horizontal rule between columns

\usepackage{tikz} % Required for flow chart
\usetikzlibrary{shapes,arrows} % Tikz libraries required for the flow chart in the template

\newcommand{\compresslist}{ % Define a command to reduce spacing within itemize/enumerate environments, this is used right after \begin{itemize} or \begin{enumerate}
\setlength{\itemsep}{1pt}
\setlength{\parskip}{0pt}
\setlength{\parsep}{0pt}
}

\definecolor{red}{rgb}{0.8,0.2,0.2} % Defines the color used for content box headers

\begin{document}

\begin{poster}
{
headerborder=closed, % Adds a border around the header of content boxes
colspacing=1em, % Column spacing
bgColorOne=white, % Background color for the gradient on the left side of the poster
bgColorTwo=white, % Background color for the gradient on the right side of the poster
borderColor=red, % Border color
headerColorOne=black, % Background color for the header in the content boxes (left side)
headerColorTwo=red, % Background color for the header in the content boxes (right side)
headerFontColor=white, % Text color for the header text in the content boxes
boxColorOne=white, % Background color of the content boxes
textborder=roundedleft, % Format of the border around content boxes, can be: none, bars, coils, triangles, rectangle, rounded, roundedsmall, roundedright or faded
eyecatcher=true, % Set to false for ignoring the left logo in the title and move the title left
headerheight=0.1\textheight, % Height of the header
headershape=roundedright, % Specify the rounded corner in the content box headers, can be: rectangle, small-rounded, roundedright, roundedleft or rounded
headerfont=\Large\bf\textsc, % Large, bold and sans serif font in the headers of content boxes
%textfont={\setlength{\parindent}{1.5em}}, % Uncomment for paragraph indentation
linewidth=2pt % Width of the border lines around content boxes
}
%----------------------------------------------------------------------------------------
%	TITLE SECTION 
%----------------------------------------------------------------------------------------
%
{\includegraphics[height=6em]{USC.png}} % First university/lab logo on the left
{\bf\textsc{Scaling blockchain cryptocurrencies with leveled zones}\vspace{0.3em}} % Poster title
{\textsc{\{ Borys Gurtovyi and Eduard Sanou \} \\ University of Southern California}} % Author names and institution
{\includegraphics[height=6em]{USC.png}} % Second university/lab logo on the right

%----------------------------------------------------------------------------------------
%	OBJECTIVES
%----------------------------------------------------------------------------------------

%\headerbox{Objectives}{name=objectives,column=0,row=0}{
%
%Donec non nisl a \textbf{arcu consequat} varius. Sed suscipit cursus luctus. Nulla sit amet elit augue. Curabitur scelerisque mollis dolor, quis blandit lorem condimentum at. Pellentesque sed nibh vel \textbf{dolor} sagittis semper. 
%
%\begin{enumerate}\compresslist
%\item Feugiat vitae elit
%\item bibendum ante sed lacinia eros in
%\item Curabitur scelerisque arcu consequat varius
%\item Dapibus nulla id purus consectetur
%\item Fringilla integer 
%\end{enumerate}
%
%\vspace{0.3em} % When there are two boxes, some whitespace may need to be added if the one on the right has more content
%}

%----------------------------------------------------------------------------------------
%	INTRODUCTION
%----------------------------------------------------------------------------------------

\headerbox{Introduction}{name=introduction,column=0,span=2,row=0}{

\begin{multicols}{2}
\vspace{1em}


Current decentralized cryptocurrencies based on Blockchains suffer from a
scalability problem that makes them unfit to replace regular fiat transactions.
We propose an idea to achieve a much higher scalability by creating transaction
zones at different levels, in which the transactions have different properties
depending on the smallest common zone-level.  A possibility would be for levels
to be: global (0), country (-1), state (-2), city (-3).  A global blockchain would
be synchronized daily (merging the blockchain of every country into one), which
every country would take as starting point to build a fork in which only
in-country transactions would be accepted.  Similarly, each country would
synchronize every 2 hours the forks of each state, and so on.  Ultimately,
in-city transactions would happen with minimal delay and only be synchronized at
state level every 30 minutes.  This would allow in-city transactions to happen
with minimal delay, in-state with 30m delay, in-country with 2h delay
and globally with 24h delay.  This design would allow the transaction throughput
within a single city to achieve the same throughput as current Blockchains
do globally.  In other words, we are optimizing transaction delays by distance,
which is a common expectation of fiat transactions, while allowing the global
transaction throughput to increase massively.

\end{multicols}

}

%----------------------------------------------------------------------------------------
%	RESULTS 2
%----------------------------------------------------------------------------------------

%\headerbox{Background}{name=results2,column=1,below=objectives,bottomaligned=conclusion}{ % This block's bottom aligns with the bottom of the conclusion block
\headerbox{Background}{name=background,column=2,span=1,row=0}{

Donec faucibus purus at tortor egestas eu fermentum dolor facilisis. Maecenas tempor dui eu neque fringilla rutrum. Mauris \emph{lobortis} nisl accumsan.

Bitcoin, Ethereum, other cryptocurrencies not scalable.  Scalable approaches are bla bla bla, and have reached this throughput bla bla.  Widely used payment systems like Visa and Paypal have higher throughput bla bla.

%\begin{center}
%\begin{tabular}{l l l}
%\toprule
%\textbf{Treatments} & \textbf{Response 1} & \textbf{Response 2}\\
%\midrule
%Treatment 1 & 0.0003262 & 0.562 \\
%Treatment 2 & 0.0015681 & 0.910 \\
%Treatment 3 & 0.0009271 & 0.296 \\
%\bottomrule
%\end{tabular}
%\captionof{table}{Table caption}
%\end{center}

Nulla ut porttitor enim. Suspendisse venenatis dui eget eros gravida tempor. Mauris feugiat elit et augue placerat ultrices. Morbi accumsan enim nec tortor consectetur non commodo.

\begin{center}
\begin{tabular}{l l}
\toprule
\textbf{Payment system} & \textbf{transactions per second} \\
\midrule
Ethereum & 20 \\
Bitcoin & 3-4 \\
Visa & 24,000 \\
Paypal &  193 \\
\bottomrule
\end{tabular}
\captionof{table}{Table caption}
\end{center}
}

%----------------------------------------------------------------------------------------
%	RESULTS 1
%----------------------------------------------------------------------------------------

\headerbox{Scaling with leveled zones}{name=main,column=0,span=2,row=1,below=introduction,above=bottom}{

\begin{multicols}{2}
\vspace{1em}
\begin{center}
\includegraphics[width=0.8\linewidth]{placeholder}
\captionof{figure}{Figure caption}
\end{center}

The global blockchain is periodically being forked into fine grained local blockchains and periodically being merged into coarser more-global blockchains.

\begin{enumerate}

    \item User assigns a wallet public key to a particular city. This information is stored in the global blockchain. However, we don't restrict him from having multiple accounts in different cities. The locality is called "city" in our system if its population exceeds 1 million people. Other smaller localities will be combined into one fork.

    \item A wallet can claim to be in a new city any time, but this information will propagate to the global (0) blockchain in the same way as transactions.

    \item Every zone creates a fork from the blockchain fork of the upper level periodically.

    \item Each zone only accepts transactions that originate from the same zone.

    \item In-zone transactions are verified during the merge of the lower level zones (or as they are collected into blocks in the city (-3) level). [add graph here]

    \item Out-zone transactions are verified when the lowest common upper level zone merges the zones it entails. [add graph here]

    \item Transactions are verified when the forks they belong to are merged into higher zones (or as they are collected into blocks in the city (-3) level).

\end{enumerate}

4 levels (from high to low) : global (0), country (-1), state (-2), city (-3)

Intervals:
\begin{itemize}
    \item Global (0): 24h
    \item Country (-1): 2h
    \item State (-2): 10 minutes
    \item City (-3): 10 seconds
\end{itemize}

When user 1 (from city A)  sends a cross-city transaction to user 2 (from city B) this transaction is being added to the city A fork as a special  “hold” transaction. Basically, user 1 cannot double-spend this money, however user 2 will not receive them until city-merge will happen.

After merge 
Merges will not have conflicts, allowing them to be fast.  Only verification against double spending is required.

If a user commutes between zones, they can have an account for each zone and split their money among the accounts.

No need to verify the actual location of the user VS the claimed location.  The claimed location only serves to decide in which local fork the account can do transactions.

%\end{multicols}

%------------------------------------------------

%\begin{multicols}{2}
%\vspace{1em}

\begin{center}
\includegraphics[width=0.8\linewidth]{placeholder}
\captionof{figure}{Figure caption}
\end{center}

\end{multicols}
}

\headerbox{Challenges}{name=challenges,column=2,span=1,below=background}{

%\begin{multicols}{2}
%\vspace{1em}

\begin{itemize}
    \item Nodes may be required to be very powerful to track the global blockchain.  How could smaller (less bandwidth/storage/computation) nodes contribute to the network?
    \item Performing a merge requires verifying all the forks against double spending.  This could require a lot of computation resources.  Could the merge output a compressed form that makes it easier to verify in merges of the next level?
    \item At the lowest level, each zone has a small number of nodes, which could make it easier for an attacker to disrupt such zone.
    \item What mitigations could be added so that an attacker doesn't disrupt a small level zone (making a denial of service).
    \item Idea: somehow, make sure that all zones have the same computing power?
    \item Merging time between forks should be faster than the interval time between merges.
\end{itemize}


%\end{multicols}
}

%----------------------------------------------------------------------------------------
%	REFERENCES
%----------------------------------------------------------------------------------------

\headerbox{References}{name=references,column=2,span=1,below=challenges,above=bottom}{

\renewcommand{\section}[2]{\vskip 0.05em} % Get rid of the default "References" section title
\nocite{*} % Insert publications even if they are not cited in the poster
\small{ % Reduce the font size in this block
\bibliographystyle{unsrt}
\bibliography{sample} % Use sample.bib as the bibliography file
}}

%----------------------------------------------------------------------------------------
%	FUTURE RESEARCH
%----------------------------------------------------------------------------------------

%\headerbox{Future Research}{name=futureresearch,column=1,span=2,aligned=references,above=bottom}{ % This block is as tall as the references block
%
%\begin{multicols}{2}
%Integer sed lectus vel mauris euismod suscipit. Praesent a est a est ultricies pellentesque. Donec tincidunt, nunc in feugiat varius, lectus lectus auctor lorem, egestas molestie risus erat ut nibh.
%
%Maecenas viverra ligula a risus blandit vel tincidunt est adipiscing. Suspendisse mollis iaculis sem, in \emph{imperdiet} orci porta vitae. Quisque id dui sed ante sollicitudin sagittis.
%\end{multicols}
%}

%----------------------------------------------------------------------------------------
%	CONTACT INFORMATION
%----------------------------------------------------------------------------------------

%\headerbox{Contact Information}{name=contact,column=2,aligned=references,above=bottom}{ % This block is as tall as the references block
%
%\begin{description}\compresslist
%\item[Web] www.university.edu/smithlab
%\item[Email] john@smith.com
%\item[Phone] +1 (000) 111 1111
%\end{description}
%}

%----------------------------------------------------------------------------------------
%	CONCLUSION
%----------------------------------------------------------------------------------------

%\headerbox{Conclusion}{name=conclusion,column=2,span=2,row=0,below=results,above=references}{
%
%\begin{multicols}{2}
%
%\tikzstyle{decision} = [diamond, draw, fill=blue!20, text width=4.5em, text badly centered, node distance=2cm, inner sep=0pt]
%\tikzstyle{block} = [rectangle, draw, fill=blue!20, text width=5em, text centered, rounded corners, minimum height=4em]
%\tikzstyle{line} = [draw, -latex']
%\tikzstyle{cloud} = [draw, ellipse, fill=red!20, node distance=3cm, minimum height=2em]
%
%\begin{tikzpicture}[node distance = 2cm, auto]
%\node [block] (init) {Initialize Model};
%\node [cloud, left of=init] (Start) {Start};
%\node [cloud, right of=init] (Start2) {Start Two};
%\node [block, below of=init] (init2) {Initialize Two};
%\node [decision, below of=init2] (End) {End};
%\path [line] (init) -- (init2);
%\path [line] (init2) -- (End);
%\path [line, dashed] (Start) -- (init);
%\path [line, dashed] (Start2) -- (init);
%\path [line, dashed] (Start2) |- (init2);
%\end{tikzpicture}
%
%%------------------------------------------------
%
%\begin{itemize}\compresslist
%\item Pellentesque eget orci eros. Fusce ultricies, tellus et pellentesque fringilla, ante massa luctus libero, quis tristique purus urna nec nibh. Phasellus fermentum rutrum elementum. Nam quis justo lectus.
%\item Vestibulum sem ante, hendrerit a gravida ac, blandit quis magna.
%\item Donec sem metus, facilisis at condimentum eget, vehicula ut massa. Morbi consequat, diam sed convallis tincidunt, arcu nunc.
%\item Nunc at convallis urna. isus ante. Pellentesque condimentum dui. Etiam sagittis purus non tellus tempor volutpat. Donec et dui non massa tristique adipiscing.
%\end{itemize}
%
%\end{multicols}
%}

%----------------------------------------------------------------------------------------
%	MATERIALS AND METHODS
%----------------------------------------------------------------------------------------

%\headerbox{Materials \& Methods}{name=method,column=0,below=objectives,bottomaligned=conclusion}{ % This block's bottom aligns with the bottom of the conclusion block
%
%The following materials were required to complete the research:
%
%\begin{itemize}\compresslist
%\item Curabitur pellentesque dignissim
%\item Eu facilisis est tempus quis
%\item Duis porta consequat lorem
%\item Eu facilisis est tempus quis
%\end{itemize}
%
%The following equations were used for statistical analysis:
%
%\begin{equation}
%\cos^3 \theta =\frac{1}{4}\cos\theta+\frac{3}{4}\cos 3\theta
%\label{eq:refname}
%\end{equation}\
%
%\begin{equation}
%E = mc^{2}
%\label{eqn:Einstein}
%\end{equation}
%
%Phasellus imperdiet, tortor vitae congue bibendum, felis enim sagittis lorem, et volutpat ante orci sagittis mi. Morbi rutrum laoreet semper. Morbi accumsan enim nec tortor consectetur non commodo nisi sollicitudin. Proin sollicitudin. Pellentesque eget orci eros. Fusce ultricies, tellus et pellentesque fringilla, ante massa luctus libero, quis tristique purus urna nec nibh.
%}
%
%----------------------------------------------------------------------------------------

\end{poster}

\end{document}
