%%%%%%%%%%%%%%%%%%%%%%%%%%%%%%%%%%%%%%%%%
% baposter Landscape Poster
% LaTeX Template
% Version 1.0 (11/06/13)
%
% baposter Class Created by:
% Brian Amberg (baposter@brian-amberg.de)
%
% This template has been downloaded from:
% http://www.LaTeXTemplates.com
%
% License:
% CC BY-NC-SA 3.0 (http://creativecommons.org/licenses/by-nc-sa/3.0/)
%
%%%%%%%%%%%%%%%%%%%%%%%%%%%%%%%%%%%%%%%%%

%----------------------------------------------------------------------------------------
%	PACKAGES AND OTHER DOCUMENT CONFIGURATIONS
%----------------------------------------------------------------------------------------

\documentclass[portrait,ansidpaper,fontscale=0.445]{baposter} % Adjust the font scale/size here

\usepackage{graphicx} % Required for including images
\graphicspath{{figures/}} % Directory in which figures are stored

\usepackage{amsmath} % For typesetting math
\usepackage{amssymb} % Adds new symbols to be used in math mode

\usepackage{booktabs} % Top and bottom rules for tables
\usepackage{enumitem} % Used to reduce itemize/enumerate spacing
\usepackage{palatino} % Use the Palatino font
\usepackage[font=small,labelfont=bf]{caption} % Required for specifying captions to tables and figures

\usepackage{multicol} % Required for multiple columns
\setlength{\columnsep}{1.5em} % Slightly increase the space between columns
\setlength{\columnseprule}{0mm} % No horizontal rule between columns

\usepackage{tikz} % Required for flow chart
\usetikzlibrary{shapes,arrows} % Tikz libraries required for the flow chart in the template

\newcommand{\compresslist}{ % Define a command to reduce spacing within itemize/enumerate environments, this is used right after \begin{itemize} or \begin{enumerate}
\setlength{\itemsep}{1pt}
\setlength{\parskip}{0pt}
\setlength{\parsep}{0pt}
}

\definecolor{red}{rgb}{0.8,0.2,0.2} % Defines the color used for content box headers

\begin{document}

\begin{poster}
{
headerborder=closed, % Adds a border around the header of content boxes
colspacing=1em, % Column spacing
bgColorOne=white, % Background color for the gradient on the left side of the poster
bgColorTwo=white, % Background color for the gradient on the right side of the poster
borderColor=red, % Border color
headerColorOne=black, % Background color for the header in the content boxes (left side)
headerColorTwo=red, % Background color for the header in the content boxes (right side)
headerFontColor=white, % Text color for the header text in the content boxes
boxColorOne=white, % Background color of the content boxes
textborder=roundedleft, % Format of the border around content boxes, can be: none, bars, coils, triangles, rectangle, rounded, roundedsmall, roundedright or faded
eyecatcher=true, % Set to false for ignoring the left logo in the title and move the title left
headerheight=0.1\textheight, % Height of the header
headershape=roundedright, % Specify the rounded corner in the content box headers, can be: rectangle, small-rounded, roundedright, roundedleft or rounded
headerfont=\Large\bf\textsc, % Large, bold and sans serif font in the headers of content boxes
%textfont={\setlength{\parindent}{1.5em}}, % Uncomment for paragraph indentation
linewidth=2pt % Width of the border lines around content boxes
}
%----------------------------------------------------------------------------------------
%	TITLE SECTION 
%----------------------------------------------------------------------------------------
%
{\includegraphics[height=6em]{USC.png}} % First university/lab logo on the left
{\bf\textsc{Scaling blockchain cryptocurrencies with leveled zones}\vspace{0.3em}} % Poster title
{\textsc{\{ Borys Gurtovyi and Eduard Sanou \} \\ University of Southern California}} % Author names and institution
{\includegraphics[height=6em]{USC.png}} % Second university/lab logo on the right

%----------------------------------------------------------------------------------------
%	INTRODUCTION
%----------------------------------------------------------------------------------------

\headerbox{Introduction}{name=introduction,column=0,span=2,row=0}{

\begin{multicols}{2}
\vspace{1em}

Current decentralized cryptocurrencies based on \textbf{blockchains suffer from a
scalability problem} that makes them unfit to replace centralized payment
systems such as VISA and PayPal.

\vspace{1em}
We propose an idea to achieve a higher scalability than current
cryptocurrencies by creating a \textbf{hierarchy of transaction zones at different
levels}, in which the transactions have different properties depending on the
smallest level shared zone.

\textbf{Each level would have a fork of a common blockchain that would be merged with
sibling zones (those under the same parent) periodically}, with the periodicity
being different at each level.  The periodicity would be lower on higher level
zones.

Transactions within the lowest level zones would be applied and verified
immediately whereas those traversing zones would only be applied once the forks of
the corresponding zones are merged.

We suggest using 4 levels: global (0), country (-1), state (-2), city (-3).

\vspace{1em}
This design would allow the transaction throughput within a single city to
achieve a similar throughput as current blockchains do globally.

In other words, we are \textbf{optimizing transaction delays by distance}, which is a
common expectation of fiat transactions, while allowing the global transaction
throughput to increase significantly.

\end{multicols}

}

%----------------------------------------------------------------------------------------
%	BACKGROUND
%----------------------------------------------------------------------------------------

%\headerbox{Background}{name=results2,column=1,below=objectives,bottomaligned=conclusion}{ % This block's bottom aligns with the bottom of the conclusion block
\headerbox{Background}{name=background,column=2,span=1,row=0}{

    \textbf{All existing blockchain protocols have a serious limitation: scalability}. It is
impossible to scale popular consensus systems such as Bitcoin because every
single node on the network processes every transaction and maintains a copy of
the entire state. The benefits of decentralization imply that every node is
independent and has equal processing requirements.

\vspace{0.5em}
With regular database system - the solution would be very
easy. We could just add more servers - the number of servers is in direct ratio
with the how system can scale. \textbf{Adding more computing power may also help but in
the decentralized world this means increasing the power of every node.}  We
cannot force every participant to do so.

\vspace{0.5em}
In fact, the blockchain actually gets weaker as more nodes are added to its
network because of the inter-node latency that logarithmically increases with
every additional node. Moreover, requirements for the node increase, and at
some point risk of centralization appears (only some part of nodes are able to
process the transaction). We end up with the trade off between throughput and
decentralization, leading to low throughput in cryptocurrencies.

\vspace{-0.5em}
\begin{center}
\begin{tabular}{l l | l l}
\textbf{System} & \textbf{tx/sec} & \textbf{System} & \textbf{tx/sec} \\
\midrule
Bitcoin & 3-4 &        Paypal &  193 \\
Ethereum & 20 &        Visa & 2,000 \\
Bitcoin Cash & 60 &          &       \\


\end{tabular}
%\captionof{table}{Table caption}
\end{center}


\vspace{-0.5em}
However this is not enough to meet the needs of fiat currency.  Nowadays,
Paypal processes 193 transactions per second and Visa 2000 (with a
capability of 24,000). \textbf{The majority of all transactions are local transaction}
(people do fiat currency transactions locally more often than to outside
places). This fact pushed us to the idea of scaling blockchain by leveled
zones.

}

%----------------------------------------------------------------------------------------
%	MAIN
%----------------------------------------------------------------------------------------

\headerbox{Scaling with leveled zones}{name=main,column=0,span=2,row=1,below=introduction, above=bottom}{

\begin{multicols}{2}

\begin{itemize}[leftmargin=1em]
    \item The global blockchain is started from the genesis block and
        is \textbf{periodically forked according to the next hierarchy}: The global
        blockchain can only be forked to a country level, so every country
        will have its own fork. The country level will be forked as well to
        state/region levels. So every state will have a fork from it's country
        chain.  Finally, the state will be forked to city levels, so every
        city will have a fork from it's state chain.

    \item A user assigns a wallet public key to a particular city\footnote{The
        locality is called "city" in our system if its population exceeds 1
        million people. Other smaller localities will be combined into one unit.}.
        This information is stored in the global blockchain.  Users don't need
        to prove their location to create a wallet - it influences only the
        fork that will validate their transactions. If a user commutes between
        zones, they can have an account for each zone and split their money
        among the accounts.

    \item A wallet can claim to be in a new city any time; this information
        will propagate to the global blockchain in the same way as
        transactions.
\end{itemize}

\textbf{All transaction in the city will be validated almost immediately} (10 seconds)
because one particular city level fork is processing only that city's
transactions. The majority of fiat transactions happen within one locality - so
that is the main goal. In this way we don't have any throughputs problems -
this is actually a sharding with a smart choice of zones.

\begin{itemize}[leftmargin=1em]
    \item When user 1 (from city A) sends a cross-city transaction to user 2
        (from city B) this transaction is being added to the city A fork as a
        special "hold" transaction. Basically, user 1 cannot double-spend this
        money, however user 2 will not verify the transaction until the proper
        merge happens.

    \item \textbf{Every 10 minutes all cities are being merged within a state}. Because
        all transactions are verified (at this moment we assume that we can
        trust other forks were not hijacked) and there is no incoming cross-city
        transactions (all transactions are either local or "held" output
        transactions), the merge can be done without any conflicts. Immediately
        after that a new fork is created for every city and the process starts
        again.

    \item \textbf{Every 2 hours all state/region forks are synced up} and merged
        to the appropriate country forks.

    \item Finally, \textbf{every 24 hours all country forks are being merged to the
    global blockchain}.

\end{itemize}

\vspace{-1.5em}
\begin{center}
\begin{tabular}{l l l}
\textbf{Name} & \textbf{Level} & \textbf{periodicity}  \\
\midrule
City  &   (-3) & 10 seconds \\
State &   (-2) & 10 minutes \\
Country & (-1) & 2 hours \\
Global  & (0)  & 24 hours \\
\end{tabular}
%\captionof{table}{Table caption}
\end{center}

All exact merging times are scheduled and unchanged.

\textbf{Each zone only accepts transactions that originate from the same zone.}
\begin{itemize}[leftmargin=1em]
    \item In-zone transactions are verified during the merge of the
        lower level zones (or as they are collected into blocks in the
        city (-3) level).
    \item Out-zone transactions are verified when the lowest common
        upper level zone merges the zones it entails.
\end{itemize}

Transactions are verified when the forks they belong to are merged into higher
zones (or as they are collected into blocks in the city (-3) level).


\end{multicols}
\begin{center}
\includegraphics[width=1\linewidth]{blockchain}
%\captionof{figure}{Figure caption}
\end{center}
}

%----------------------------------------------------------------------------------------
%	CHALLENGES
%----------------------------------------------------------------------------------------

\headerbox{Challenges}{name=challenges,column=2,span=1,below=background,above=bottom}{

\begin{itemize}[leftmargin=1em]
    \item \textbf{The merging process requires more computational work and bandwidth on
        higher levels}.  Not all nodes may be capable enough to verify merges at
        every level,  Flexibility in the amount of contribution by node may be
        required.
    \item \textbf{When a node performs a merge, it needs to verify the forks of the
        sibling zones for correctness}.  This could require a lot of computation
        resources.  A possible optimization would be that each merge outputs a
        compressed form of the result that makes verification at the next level
        faster.
    \item \textbf{At the lowest level, each zone would have a small number of nodes}.
        This opens the possibility of an attacker disrupting the zone by
        registering nodes in it with high validation weight (either
        computational power or stake).  Such disruption could be a denial of
        service or a denial of selected transactions.
    \begin{itemize}[leftmargin=1em]
        \item A way to make make sure that all zones have the same computing
            power could be useful to solve this.
        \item A different approach would consist on restricting how easily new
            nodes with high computational power can join or switch zones.
    \end{itemize}
    \item \textbf{At each level, the merging time between forks should be faster than the
        periodicity of merges}.
\end{itemize}

}

%----------------------------------------------------------------------------------------
%	REFERENCES
%----------------------------------------------------------------------------------------

%\headerbox{References}{name=references,column=0,span=2,below=main,above=bottom}{
%
%\renewcommand{\section}[2]{\vskip 0.05em} % Get rid of the default "References" section title
%\nocite{*} % Insert publications even if they are not cited in the poster
%\small{ % Reduce the font size in this block
%\bibliographystyle{unsrt}
%\bibliography{sample} % Use sample.bib as the bibliography file
%}}

\end{poster}

\end{document}
